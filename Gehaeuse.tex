% !TEX root = SegwayDoku.tex
\renewcommand{\autoren}{Valentyn Chepil, Alexsander Stoiljkovic}
\newpage
\section{Die Gehäuse}
\subsection{Die Gehäusen - V.1}
% \ref{bild_3} zuweisung auf Bild in Text.

\begin{figure}[!h]  % [h] bedeutet, dass das Bild genau an dieser Stelle im Text erscheint
	% mit width=... wird die Größe des Bildes in Prozent der Seitenbreite eingestellt
	\centering\includegraphics[width=0.5\textwidth]{images/gehaeuse-v1.png}
	% caption ist die Bildunterschrift, taucht auch im Abbildungsverzeichnis auf
	\caption{Gehäuse - V.1 \newline (Quelle: eigene Darstellung)}
	\label{gehaeuse-v1} % über das label kann man aus dem Text auf das Bild verweisen
\end{figure}

Es wurde gesagt, sehr einfachen Aufbau des Gehäuse im Form des Kasten herzustellen.


\subsection{Die Gehäusen - V.2}


\begin{figure}[!h]  % [h] bedeutet, dass das Bild genau an dieser Stelle im Text erscheint
	% mit width=... wird die Größe des Bildes in Prozent der Seitenbreite eingestellt
	\centering\includegraphics[width=0.5\textwidth]{images/gehaeuse-v2.png}
	% caption ist die Bildunterschrift, taucht auch im Abbildungsverzeichnis auf
	\caption{Gehäuse - V.2 \newline (Quelle: eigene Darstellung)}
	\label{gehaeuse-v2} % über das label kann man aus dem Text auf das Bild verweisen
\end{figure}


Im Bild \ref{gehaeuse-v2} ist eine modulare Darstellung des Roboters zu sehen. Das Motorhalter ist in diesem Fall als ein externer Modul dergestalt. Im ersten Modul von oben sollte man alle Steuerungsplatinen  einbauen. Das zweiten Modul von Oben beinhaltete das Akku


\renewcommand{\autoren}{Valentyn Chepil}
\newpage

\subsection{Die Gehäusen - V.3}


\begin{figure}[!h]  % [h] bedeutet, dass das Bild genau an dieser Stelle im Text erscheint
	% mit width=... wird die Größe des Bildes in Prozent der Seitenbreite eingestellt
	\centering\includegraphics[width=0.5\textwidth]{images/gehaeuse-v3.png}
	% caption ist die Bildunterschrift, taucht auch im Abbildungsverzeichnis auf
	\caption{Gehäuse - V.2 \newline (Quelle: eigene Darstellung)}
	\label{gehaeuse-v3} % über das label kann man aus dem Text auf das Bild verweisen
\end{figure}

\subsection{Die Gehäusen - V.4}


%\begin{figure}[!h]  % [h] bedeutet, dass das Bild genau an dieser Stelle im Text erscheint
	% mit width=... wird die Größe des Bildes in Prozent der Seitenbreite eingestellt
%	\centering\includegraphics[width=0.5\textwidth]{images/gehaeuse-v4.png}
	% caption ist die Bildunterschrift, taucht auch im Abbildungsverzeichnis auf
%	\caption{Gehäuse - V.4 \newline (Quelle: eigene Darstellung)}
%	\label{gehaeuse-v4} % über das label kann man aus dem Text auf das Bild verweisen
%\end{figure}





\subsubsection{ Berechnung vom Motorhalter}

Alle Berechnungen wurden mit Hilfe von FEM - Programmierung gemacht und beweise die Festigkeit nur von dem Motorhalter. Es wurden an den Stellen vom Motoren und Räder das Moment laut der Berechnung %\ref{ber}
 verwendet und als Lastkraft wurde 50N auf die Oberfläche eingeprägt. 

Bild einfügen mit %\ref{ber}

Die Materialeigenschaften kann man auf der Abbildung \ref{FEM2}

\begin{figure}[!h]  % [h] bedeutet, dass das Bild genau an dieser Stelle im Text erscheint
	\centering\includegraphics[width=0.7\textwidth]{images/FEM2.png}
	% caption ist die Bildunterschrift, taucht auch im Abbildungsverzeichnis auf
	\caption{Kunststoffeigenschaften - Nylon (PA) \newline (Quelle: eigene Darstellung)}
	\label{FEM2} % über das label kann man aus dem Text auf das Bild verweisen
\end{figure}

\begin{figure}[!h]  % [h] bedeutet, dass das Bild genau an dieser Stelle im Text erscheint
	% mit width=... wird die Größe des Bildes in Prozent der Seitenbreite eingestellt
	\centering\includegraphics[width=0.9\textwidth]{images/FEM.png}
	% caption ist die Bildunterschrift, taucht auch im Abbildungsverzeichnis auf
	\caption{FEM-Berechnung \newline (Quelle: eigene Darstellung)}
	\label{FEM1} % über das label kann man aus dem Text auf das Bild verweisen
\end{figure}



\begin{figure}[!h]  % [h] bedeutet, dass das Bild genau an dieser Stelle im Text erscheint
	\centering\includegraphics[width=0.9\textwidth]{images/FEM3.png}
	% caption ist die Bildunterschrift, taucht auch im Abbildungsverzeichnis auf
	\caption{FEM - Verschiebungen \newline (Quelle: eigene Darstellung)}
	\label{FEM3} % über das label kann man aus dem Text auf das Bild verweisen
\end{figure}
\begin{figure}[!h]  % [h] bedeutet, dass das Bild genau an dieser Stelle im Text erscheint
	\centering\includegraphics[width=0.9\textwidth]{images/FEM4.png}
	% caption ist die Bildunterschrift, taucht auch im Abbildungsverzeichnis auf
	\caption{FEM - Spannungen \newline (Quelle: eigene Darstellung)}
	\label{FEM4} % über das label kann man aus dem Text auf das Bild verweisen
\end{figure}