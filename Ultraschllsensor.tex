% !TEX root = SegwayDoku.tex
\renewcommand{\autoren}{Valentyn Chepil}
\newpage
\section{Die Hinderniserkennung}

Die Erkennung vom Hindernis erfolgt über ausgehenden Schallwellen, die sich in kegelförmig Form  ausbreiten. Bei Erreichung des Ziels werden die reflektierende Schallwellen aufgenommen und das Zeig des Reflektion wird ausgemessen. Dadurch erkennt man das Hindernis und den Abstand zu dem. Die Ausbreitung und die reflektieren Schallwellen hängt von verschiedenen Faktoren
ab. Sie werden vom Luftdruck, der Umgebungstemperatur, der Luftfeuchtigkeit sowie
vom Sende-/Empfangswinkel und der Oberfläche des im Schallkegel befindlichen Objekts
beeinflusst.

  Vorteile:


Ein größter Vorteil des Ultraschallsensors ist günstiger Preis. Dies ermöglicht die Verwendung von vielen Sensoren bei der Modellbau, was zur präzisen Erfassung der Entfernung zu Hindernissen und Objekten in der Umgebung führt.

//bild//

  Nachteile:


Ein Hindernis oder Objekt kann nicht genau geortet werden, sondern man kann nur feststellen, dass es sich in der gemessenen Entfernung innerhalb des Schallkegels befindet. 

//bild//

Bei mehrere Objekten vor Schallwelle wird der Abstand zum nächstliegenden Hindernis gemessen und die weit liegender Objekte werde einfach nicht erkannt.

//bild//

Als Nachteil kann man nennen die geringe Reichweite (ca. 3m) des Ultraschalls und  der geringen Ausbreitungsgeschwindigkeit des Schalls.

//bild//

Große Probleme bekommt man auch aus den Reflektionseigenschaften der Schallwellen.


//bild//

Wird das Winkel des Schallkegels größer seinem Öffnungswinkel auf eine ebene Wand, so erreicht die Reflektion den aussendenden Sensor nicht.

//bild//

 Oberflächenerkennung erfasst das Sensor auch schleicht.


//bilder//
