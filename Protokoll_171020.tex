\documentclass[10pt]{scrartcl}
\usepackage[german,ngerman]{babel}
\usepackage[utf8]{inputenc}
\usepackage[T1]{fontenc}
\usepackage{tabularx}

\title{Protokoll zur Teambesprechung}
\date{20.10.2017}%Ändern


\begin{document}
\maketitle
\thispagestyle{empty}

\begin{tabularx}{\textwidth}{lp{9 cm}} 
Anwesend: & Prof. Nitzsche, Chepil Valentyn, Stoiljkovic Aleksandar, Morongowski Stephan,  Veit Timo, Schendel Severin\\% Ändern bei Bedarf
Abwesend: & - \\
Protokollführer: & Severin Schendel \\
Projektleiter: & Prof. Nitzsche\\
Nächstes Meeting: & 20.10.17 \\%Ändern
&\\
\end{tabularx}

\begin{enumerate}
	\item Angesetzte Themen
		\begin{enumerate}
			\item Besprechung des Zeitplans.% Ab hier neue Themen eintragen.
		\end{enumerate}
	\item Beschlüsse
		\begin{enumerate}
			\item Zeitplan ist inhaltlich unvollständig. Umbenenung der Meilensteine% Ab hier neue Beschlüsse eintragen.
			\item Die großen Ziele festlegen.
			\begin{itemize}
				\item Umbau auf Direktantrieb
				\item Hinderniserkennung
				\item Gehäuseumbau
				\item Kurvenfahrt $\rightarrow$ Regler für Kurven und Geschwindigkeit
				\item Verbesserung der Kippwinkelerkennung, zusätzlich Modellierung des Sensors
			\end{itemize}
			\item Versuchsaufbauten für einzelne Komponenten.
			\item Einen vordefinierten Gegenstand erkennen ist das Minimalziel der Hinderniserkennung.
			\item Andere Möglichkeiten zur Realisierung des Direktantriebes finden, falls BLDC zu kompliziert oder teuer.
		\end{enumerate}

	\end{enumerate}




\end{document}