% !TEX root = Beispielhauptdokument.tex

% in der ersten Zeile steht ein "magischer Kommentar". Dadurch weiß das Editierprogramm immer, welche Datei die Hauptdatei ist.

% die folgenden Zeilen werden benötigt, damit der richtige Author in der Fußzeile der Seite auftaucht.
\renewcommand{\autoren}{Stephan Morongowski}
\newpage

\section{eine Überschrift}

% Das Zitat mit der Referenz "bildungAtmosphaere" muss im Quellenverzeichnis stehen. Das kann zB mit JabRef bearbeitet werden oder einfach per Hand.
Hier ein Beispiel für ein Zitat: Wie in \cite{bildungAtmosphaere} beschrieben, ... das steht dann im Literaturverzeichnis

\subsection{eine Unterüberschrift}
eine kleine Aufzählung:
\begin{itemize}
\item Punkt 1
\item Punkt 2
\end{itemize}

Wie in Abbildung \ref{kurvenkinematik} zu sehen ist,...

\begin{figure}[h]  % [h] bedeutet, dass das Bild genau an dieser Stelle im Text erscheint
% mit width=... wird die Größe des Bildes in Prozent der Seitenbreite eingestellt
\centering\includegraphics[width=0.5\textwidth]{images/Kurvenkinematic.eps}
% caption ist die Bildunterschrift, taucht auch im Abbildungsverzeichnis auf
\caption{Rotation um den Momentanpol \newline (Quelle: eigene Darstellung)}
\label{kurvenkinematik} % über das label kann man aus dem Text auf das Bild verweisen
\end{figure}

Mathematische Sonderzeichen wie \(v_1\) oder \(\mu\) können so auch im Text eingegeben werden. Nennt sich inline-math-mode.

Jetzt kommen ein paar Matheformeln:
\begin{flalign}
    % durch das & Zeichen werden alle Gleichungen an diesem Punkt ausgerichtet
	arc_{R1} &  = \Delta\gamma\cdot r_{R1}
	\label{eq:bogenmaß_1} \\
	r_{R1} & = r_{R2}  + \prescript{0}{2}{T}_1^2
	\label{eq:achsZuMomentanpol}     % hier keine \\ einbauen, sonst kommt eine Formelnummer zuviel.
\end{flalign}

So verweist man auf Formeln: Aus \eqref{eq:bogenmaß_1} und \eqref{eq:achsZuMomentanpol} folgt zB diese ganz abgefahrene Formel mit der Nummer \eqref{eq:S_1}: % hier keine Leerzeile machen, sonst wird der Abstand ganz groß
\begin{flalign}
	\vec{S_1} = \vec{S_0} +
        \begin{pmatrix}
            \cos{(\Delta\gamma)} & -\sin{(\Delta\gamma)}  \\
            \sin{(\Delta\gamma)} & \cos{(\Delta\gamma)}
        \end{pmatrix}
	\label{eq:S_1}
\end{flalign}



\newpage
So kann das Listing \ref{lst:deadReackon}, also ein Codebeispiel, im Text eingebettet werden:
\begin{lstlisting}[language=C++, caption=deadReckonTotalPhi, label={lst:deadReackon}]
float deadReckonTotalPhi(int inkLeft, int inkRight) {
	static float totalGamma = 0.0f;
    static int fullTurns = 0;
	totalGamma = totalGamma + 1/l_a * (inkRight - inkLeft);
	return totalGamma;
}
\end{lstlisting}

\par\bigskip

Hier eine selbsterstellte Umgebung, mit der man eine 2-spaltige Tabelle mit Titel machen kann:
\par\bigskip
% erstellt eine zweispaltige Tabelle mit fett gedruckter Überschrift
\begin{benannteAuflistung}[Gewünschte Technische Daten:]
    Motorspannung & bis 25V \\
    Strom & dauerhaft 30A, Spitze 50A \\
    Schnittstellen & I2C, USB, PWM \\
\end{benannteAuflistung}

\par\bigskip
Es können auch mehr Spalten definiert werden wie im tabularx-Paket (siehe zweites Argument [llX])
\par\bigskip
\begin{benannteAuflistung}[Gewünschte Technische Daten:][llX]
    Motorspannung & bis 25V & 10,- €\\
    Strom & dauerhaft 30A, Spitze 50A & 5,- € \\
    Schnittstellen & I2C, USB, PWM & 3,- €\\
    \textbf{Summe} &  & \textbf{300,- €}\\
\end{benannteAuflistung}


% das hier muss ganz ans Ende eines Unterdokumentes, damit die Autorangabe in der Fußzeile wieder stimmt.
\newpage
