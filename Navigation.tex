% !TEX root = SegwayDoku.tex
\renewcommand{\autoren}{Stephan Morongowski}
\newpage
\section{Navigation}
\subsection{Sensorik}
Zur Verfügung stehen folgende Sensoren:
\begin{itemize}
\item Gyros in allen drei Raumachsen
\item Beschleunigungssensoren in allen drei Raumachsen
\item Inkrementalgeber der Räder
\item Ultraschallsensoren
\end{itemize}

\subsection{Inkrementalgeber}

Zur Bestimmung der Position des Roboters in einem festen Koordinatensystem können die Inkrementalgeber der Antriebsmotoren benutzt werden.

\begin{figure}[h]  % [h] bedeutet, dass das Bild genau an dieser Stelle im Text erscheint
\centering\includegraphics[width=0.8\textwidth]{images/Kurvenkinematic.eps}
\caption{Rotation um den Momentanpol \newline (Quelle: eigene Darstellung, SM)}
\label{kurvenkinematik}
\end{figure}

Zur Ermittlung der aktuellen Position ist in Abbildung \ref{kurvenkinematik} die Kinematik einer Kurvenfahrt dargestellt, betrachtet für einen gedanklich sehr kleinen Zeitabschnitt. In diesem wird angenommen, dass die Geschwindigkeiten \(v_1\) und \(v_2\) beider Räder konstant bleiben. Dies sorgt für die Vereinfachung, dass der Roboter sich auf einer Kreisbahn um einen für die betrachtete Zeitspanne konstanten Momentanpol bewegt. Im Folgenden sind die durch den Inkrementalgeber erfassten Bogenlängen mit arc bezeichnet.
Es gilt:
\begin{flalign}
    % durch das & Zeichen werden alle Gleichungen an diesem Punkt ausgerichtet
	arc_{R1} &  = \Delta\gamma\cdot r_{R1}
	\label{eq:bogenmaß_1} \\
	arc_{R2} & = \Delta\gamma\cdot r_{R2}
	\label{eq:bogenmaß_2} \\
	r_{R2} & = r_{R1}  + l_a
	\label{eq:achsZuMomentanpol} \\
	\Delta arcs & = arc_{R2} - arc_{R1}
	\label{eq:arcsDef} \\
    r_S & = r_{R2} - \frac{1}{2} l_a
	\label{eq:r_s}
\end{flalign}

Aus \eqref{eq:bogenmaß_1}, \eqref{eq:bogenmaß_2}, \eqref{eq:achsZuMomentanpol}, \eqref{eq:arcsDef} und \eqref{eq:r_s} folgen:
% hier keine Leerzeile machen, sonst wird der Abstand ganz groß
\begin{flalign}
    \Delta\gamma & = \frac{1}{l_a} \cdot \Delta arcs
	\label{eq:deltaPhi} \\
    r_S & = \frac{arc_{R2}}{\Delta\gamma} - \frac{1}{2} l_a
\end{flalign}

Timo Veit:
Die Verschiebungen aus den Geschwindigkeiten nenne ich erstmal $x_*$.
%In Simulink verwende ich erstmal die Geschwindigkeiten und integriere dann danach, um auf die richtige Einheit zu kommen. Da aus Geschwindigkeit mal Zeit wieder Strecke wird funktioniert das auch.

\begin{flalign}
	x_S &= \frac{x_2-x_1}{2}+x_1	\\
	r_S &=  \frac{l_a}{ \frac{x_2}{x_1}-1} +  \frac{l_a}{2}\\
	\Delta\gamma &= atan( \frac{x_S}{r_S})\\
	\vec{S_1} &= \vec{S_0} +
        \begin{pmatrix}
            r_S [\cos{(\gamma_0 + \Delta\gamma)}]  \\
            r_S [\sin{(\gamma_0 + \Delta\gamma)}]
        \end{pmatrix}	
\end{flalign}
Durch Aufsummieren von \(\Delta\gamma\) nach jeder Auswertung der Inkrementalgeber kann somit ein ungefährer Absolutwinkel der Roboterachse zu einem festen KS berechnet werden.
\par\bigskip
Zur Bestimmung des Schwerpunktes in xy-Koordinaten wird die Verschiebung von
\(S_0\) zu \(S_1\) mit trigonometrischen Funktionen berechnet und anschließend aufsummiert. Es gilt:
\begin{flalign}
	\vec{S_1} = \vec{S_0} +
        \begin{pmatrix}
            -\sin{(\Delta \gamma)} \cdot r_S \cdot \sin{(\gamma_0)}
            - (r_S - \cos{(\Delta \gamma)} \cdot r_S) \cdot \cos{(\gamma_0}) \\
            \sin{(\Delta \gamma)} \cdot r_S \cdot \cos{(\gamma_0)}
            - (r_S - \cos{(\Delta \gamma)} \cdot r_S) \cdot \sin{(\gamma_0})
        \end{pmatrix}
	\label{eq:S_1}
\end{flalign}

bzw. vereinfacht:
\begin{flalign}
	\vec{S_1} = \vec{S_0} +
        \begin{pmatrix}
            r_S [\cos{(\gamma_0 + \Delta\gamma)} - \cos{\gamma_0}]  \\
            r_S [\sin{(\gamma_0 + \Delta\gamma)} - \sin{\gamma_0}]
        \end{pmatrix}
	\label{eq:S_1_easy}
\end{flalign}






\newpage
